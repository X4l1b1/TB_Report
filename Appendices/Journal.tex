% Chapter 1

\appendix{Journal de Travail} % Main chapter title

\label{journal} % For referencing the chapter elsewhere, use \ref{Chapter1} 
 \addcontentsline{toc}{chapter}{Annexe A - Journal}
%----------------------------------------------------------------------------------------

%----------------------------------------------------------------------------------------

\section*{23.02.2019}

\begin{description}
	\item [Séance avec M.Thoma et M.Wortenbroek -] Explication plus détaillée du contexte du travail et du travail demandé. Présentation d'un plan initial de déroulement et définition d'une première étape, à savoir implémenter dans un language de haut niveau un programme permettant d'appliquer la \textrm{BWT} et l'indexage \textrm{FM-Index} à des entrées en format \textrm{FASTA}.
	\item [Installation -] récupération du dépôt \textrm{GitLab} et mise en place de la structure de ce dernier. Début d'ébauche d'un cahier des charges et de rapport. Mise en place de l'environnement sur ma machine.
\end{description}

\section*{28.02.2018}

\begin{description}
	\item [Cahier des charges] Complétion d'un cahier des charges un peu formel à faire valider par le professeur.
	\item [Documentation] Récupération de documents (cours, ...) relatifs à la BWT et à l'indexage FM.
	
\end{description}

\section*{02.03.2018}

\begin{description}
	\item [Cahier des charges] Finalisation du cahier des charges
	\item [Documentation] Lecture et prise de note sur les principes de transformation et d'indexage
	\item [Rapport] Début de la rédaction de l'introduction du rapport.
	
\end{description}

\section*{09.03.2018}

\begin{description}
	\item [Cahier des charges] Dernière correction du cahier des charges.
	\item [Rapport] Continuation de la rédaction de l'introduction du rapport.
	\item [Software] Début d'implémentation du software (objectif \textbf{[2]}
\end{description}