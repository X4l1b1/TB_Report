% Chapter Template

\chapter{Hardware Implementation of the FM-Index} % Main chapter title

\label{Chapter3} % Change X to a consecutive number; for referencing this chapter elsewhere, use \ref{ChapterX}			
The main scope of this thesis is to implement the aforementioned algorithm on a specific \textsl{FPGA} board, using a special type of memory to store the reference data, called \textsl{Hybrid Memory Cube} (HMC). Further explained in the next section, the idea is to make the best out of both those tools' main advantages, a high capacity for parallelization.


\section{Tools Introduction}

\subsection{Hybrid Memory Cube and Micron AC-510}


\textsl{Hybrid Memory Cube} (HMC) is a high-performance \textsl{RAM} interface for stacked \textsl{DRAM} memory. Combining \textsc{through-silicon vias} and \textsl{microbumps} (WIKIPEDIA) to connect multiple layers of memory cells on top of each others, it offers very high throughput parallel serial bus' for random i/o's. In the scope of this project, it is an ideal candidate as to where to store all the references for the  FM-Index string matching algorithm as a parallel solution would indeed query for information potentially dispatched all over the memory.


\subsection{Project Architecture}

This project is implemented in VHDL and included in a pre-existing project, the \textsl{Pico-Base Project}. [je ne sais pas trop quoi dire ...]

\section{Specifications}

[pareil ici, parler des entrées sorties, etc,.. un peu comme les specs des labos mais sans les contraintes de timing ?]

\section{Model Conception}

A first, non parallel, model conception is presented in the Figures

\begin{figure}
    \centering
    \includegraphics{}
    \caption{Caption}
    \label{fig:my_label}
\end{figure}


\section{VHDL Implementation}


\section{Test \& Validation}













