% Chapter Template

\chapter{Hardware Implementation of the FM-Index} % Main chapter title

\label{Chapter3} % Change X to a consecutive number; for referencing this chapter elsewhere, use \ref{ChapterX}			
The main scope of this thesis is to implement the aforementioned algorithm on a specific \textsl{FPGA} board, using a special type of memory to store the reference data, called \textsl{Hybrid Memory Cube} (HMC). Our hope is to attain a high parallelization factor, improving greatly performances over a large set of query. The different hardware tools used in this project will be described in this section.

\section{Tools Introduction}

\subsection{Hybrid Memory Cube and Micron AC-510}


\textsl{Hybrid Memory Cube} (HMC) is a high-performance \textsl{RAM} interface for stacked \textsl{DRAM} memory. Combining \textsc{through-silicon vias} and \textsl{microbumps} (WIKIPEDIAA) to connect multiple layers of memory cells on top of each others, it offers very high throughput parallel serial bus' for i/o.

FAIRE UNE MEILLEURE DESCRIPTION 

DECRIRE LA BOAR, PARLER DE PCI-EXPRESS AUSSI ET DE L'API

\subsection{Vivado Project}

\section{Specifications}


\section{Model Conception}

\section{VHDL Implementation}


\section{Test \& Validation}













