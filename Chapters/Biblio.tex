% Chapter Template

\addchaptertocentry{Bibliography}

\label{Biblio} % Change X to a consecutive number; for referencing this chapter elsewhere, use \ref{ChapterX}

%----------------------------------------------------------------------------------------
%	SECTION 1
%----------------------------------------------------------------------------------------
\begin{thebibliography}{9}

\bibitem{lamport94}
 Bashford, J. N., Norwood, J., \& Chapman, S. R. (1998). Why are patients prescribed proton pump inhibitors? Retrospective analysis of link between morbidity and prescribing in the General Practice Research Database. \textit{British Medical Journal, 317}(7156), 452-456. 
  
  \bibitem{lamport93}
 Bourne, C., Charpiat, B., Charhon, N., Bertin, C., Gouraud, A., Mouchoux, C., \& Janoly-Dumenil, A. (2013). Effets indésirables émergents des inhibiteurs de la pompe à protons. \textit{ Presse Médicale, 42}(2), e53-e62. 
  
  \bibitem{lamport92}
  Cunningham, R., Dale, B., Undy, B., \& Gaunt, N. (2003). Proton pump inhibitors as a risk factor for Clostridium difficile diarrhoea. \textit{Journal of Hospital Infection, 54}(3), 243-245. 
  
  \bibitem{lamport91}
  Dial, S., Alrasadi, K., Manoukian, C., Huang, A., \& Menzies, D. (2004). Risk of Clostridium difficile diarrhea among hospital inpatients prescribed proton pump inhibitors: cohort and case–control studies. \textit{Canadian Medical Association Journal, 171}(1), 33-38. 
  
  \bibitem{lamport90}
  Festinger, L. (1950). Informal social communication. \textit{Psychological review, 57}(5), 271-282. 
  
\bibitem{lamport89}
  Festinger, L. (1954). A theory of social comparison processes. \textit{Human relations, 7}(2), 117-140. 
  
  \bibitem{lamport88}
 Forgacs, I., \& Loganayagam, A. (2008). Overprescribing proton pump inhibitors. \textit{British Medical Journal, 28}(7634), 2-3. 
  
  \bibitem{lamport87}
  Leslie Lamport,
  \textit{\LaTeX: a document preparation system},
  Addison Wesley, Massachusetts,
  2nd edition,
  1994.
  
  \bibitem{lamport941}
 Freidson, E. (1984). \textit{La Profession Médicale}. Paris : Payot. 
  
  \bibitem{lamport942}
  Giger, J.-C. (2008). Examen critique du caractère prédictif, causal et falsifiable de deux théories de la relation attitude-comportement: la théorie de l’action raisonnée et la théorie du comportement planifié. \textit{L'année Psychologique, 108}(1), 107-131. 
  
  \bibitem{lamport924}
  Giuliano, C., Wilhelm, S. M., \& Kale-Pradhan, P. B. (2012). Are proton pump inhibitors associated with the development of community-acquired pneumonia? A meta-analysis.\textit{ Expert review of clinical pharmacology, 5}(3), 337-344. 
  
\bibitem{lamport934}
  Gulmez, S. E., Holm, A., Frederiksen, H., Jensen, T. G., \& Hallas, J. (2007). Use of proton pump inhibitors and the risk of community-acquired pneumonia: a population-based case-control study. \textit{Archives of Internal Medicine, 167}(9), 950-955. 
  
  \bibitem{lamport944}
  Haenisch, B., von Holt, K., Wiese, B., Prokein, J., Lange, C., Ernst, A., . . . Weyerer, S. (2015). Risk of dementia in elderly patients with the use of proton pump inhibitors.\textit{European archives of psychiatry and clinical neuroscience, 265}(5), 419-428.
  
  \bibitem{lamport954}
 Hallsworth, M., Chadborn, T., Sallis, A., Sanders, M., Berry, D., Greaves, F., \& Davies, S. C. (2016). Provision of social norm feedback to high prescribers of antibiotics in general practice: a pragmatic national randomised controlled trial. \textit{The Lancet, 387}(10029), 1743-1752. 
  
  \bibitem{lamport794}
  Heidelbaugh, J. J., Goldberg, K. L., \& Inadomi, J. M. (2009). Overutilization of proton pump inhibitors: a review of cost-effectiveness and risk in PPI. \textit{The American journal of gastroenterology, 104}, S27-S32. 
  
  \bibitem{lamport9334}
  Herghelegiu, A. M., Prada, G. I., \& Nacu, R. (2016). Prolonged use of proton pomp inhibitors and cognitive function in older adults. \textit{Farmacia Hospitalaria, 64}(2), 262-267. 
  
  \bibitem{lamport1294}
 Herranz, M. P. (2016). Is there an overprescription of proton pump inhibitors in oncohematologic patients undergoing ambulatory oncospecific treatment? \textit{Farmacia Hospitalaria, 40}(5), 436-446. 
 
\bibitem{lamport9554}
 Katz, J. (1984). Why doctors don't disclose uncertainty. \textit{Hastings Center Report, 14}(1), 35-44. 
  
  \bibitem{lampor33t94}
 Lam, J. R., Schneider, J. L., Zhao, W., \& Corley, D. A. (2013). Proton pump inhibitor and histamine 2 receptor antagonist use and vitamin B12 deficiency. \textit{Jama, 310}(22), 2435-2442. 
  
  \bibitem{lamport4594}
  Leyens, J.-P., \& Yzerbyt, V. (1997). \textit{Psychologie sociale}. Bruxelles : Editions Mardaga.
  
  \bibitem{lamport9er4}
 Maffei, M., Desmeules, J., Cereda, J., \& Hadengue, A. (2007). Effets indésirables des inhibiteurs de la pompe à protons (IPP). \textit{Revue médicale suisse, 3}(123), 1934-1938. 
  
  \bibitem{lampoet94}
 Martin, R. M., Dunn, N. R., Freemantle, S., \& Shakir, S. (2000). The rates of common adverse events reported during treatment with proton pump inhibitors used in general practice in England: cohort studies. \textit{British journal of clinical pharmacology, 50}(4), 366-372. 
  
  \bibitem{lamp324ort94}
  Nouguez, E. (2007). La définition des médicaments génériques entre enjeux thérapeutiques et économiques: L'exemple du marché français des inhibiteurs de la pompe à protons. \textit{Revue française des affaires sociales}, 3, 99-121. 
  
\bibitem{lawwmport94}
  Pollock, K., \& Grime, J. (2003). The cost and cost-effectiveness of PPIs: GP perspectives and responses to a prescribing dilemma and their implications for the development of patient-centred healthcare. \textit{The European journal of general practice, 9}(4), 126-133. 
  
  \bibitem{lampewort94}
 Reimer, C., Søndergaard, B., Hilsted, L., \& Bytzer, P. (2009). Proton-pump inhibitor therapy induces acid-related symptoms in healthy volunteers after withdrawal of therapy. \textit{Gastroenterology, 137}(1), 80-87. 
  
  \bibitem{laddmport94}
  Reinberg, O. (2015). Inhibiteurs de la pompe à protons (IPP): peut-être pas si inoffensifs que cela. \textit{Revue Medicale Suisse, 485}(11), 1665-1671. 
  
  \bibitem{laeemport94}
  Roulet, L., Vernaz, N., Giostra, E., Gasche, Y., \& Desmeules, J. (2012). Effets indésirables des inhibiteurs de la pompe à proton: faut-il craindre de les prescrire au long cours? \textit{La Revue de médecine interne, 33}(8), 439-445. 
  
  \bibitem{lampogtrt94}
  Sanduleanu, S., Stridsberg, M., Jonkers, D., Hameeteman, W., Biemond, I., Lundqvist, G., \& Stockbrügger, R. (1999). Serum gastrin and chromogranin A during medium-and long-term acid suppressive therapy: a case-control study. \textit{Alimentary Pharmacology and Therapeutics, 13}(2), 145-154. 
  
  \bibitem{ldemport94}
  Steuerwald, M., \& Meier, R. (2003). Troubles peptiques (2e partie): Ulcères peptiques. \textit{Forum Medical Suisse, 3}(42), 1008-1012. 
  
  \bibitem{lampfrort94}
  Thomson, A. B., Sauve, M. D., Kassam, N., \& Kamitakahara, H. (2010). Safety of the long-term use of proton pump inhibitors. \textit{World Journal of Gastroenterology, 16}(19), 2323-2330
  
  \bibitem{lampotgrt94}
  Turner, J. C. (1991). \textit{Social influence}. Belmont, CA : Thomson Brooks/Cole Publishing Co.
  
  \bibitem{lamposxrt94}
  Van Soest, E., Siersema, P., Dieleman, J., Sturkenboom, M., \& Kuipers, E. (2006). Persistence and adherence to proton pump inhibitors in daily clinical practice. \textit{Alimentary pharmacology \& therapeutics, 24}(2), 377-385. 
  
  \bibitem{lampgbort94}
  Waldum, H., Arnestad, J., Brenna, E., Eide, I., Syversen, U., \& Sandvik, A. (1996). Marked increase in gastric acid secretory capacity after omeprazole treatment. \textit{Gut, 39}(5), 649-653. 
  
  \bibitem{laolmport94}
  Waldum, H. L., Sandvik, A. K., Syversen, U., \& Brenna, E. (1993). The Enterochromaffin-Like (ECL) Cell Physiological and pathophysiological role. \textit{Acta Oncologica, 32}(2), 141-147. 
  
  \bibitem{lampornht94}
  Walker, N., \& McDonald, J. (2001). An evaluation of the use of proton pump inhibitors. \textit{Pharmacy World and Science, 23}(3), 116-117.
  
    \bibitem{laoort94}
 Weill, C. (1990). Attitudes professionnelles et diffusion de la connaissance scientifique: les conférences de consensus sont-elles susceptibles de modifier les comportements des praticiens? \textit{Sciences sociales et santé, 8}(4), 91-114.  
  
  \bibitem{lampornhpt94}
 Yang, Y.-X., Lewis, J. D., Epstein, S., \& Metz, D. C. (2006). Long-term proton pump inhibitor therapy and risk of hip fracture. \textit{Jama, 296}(24), 2947-2953. \\
\vspace*{1cm} \\
{\huge{\textbf{Sites Internet}}}


	\bibitem{sdkljg}
    Dictionnaire médical de l'Académie de Médecine– version 2016-1. Consulté à http://dictionnaire.academie-medecine.fr/
    
    \bibitem{sdf}
    Les figures 1-3 ont été reproduites à partir de http://bruluresdestomac.e-monsite.com. 
    
    \bibitem{dklsfjsdk}
    Compendium: consulté à https://compendium.ch/ \\   
\vspace*{1cm} \\
{\huge{\textbf{Vidéo}}}

\bibitem{kldsfjds}
Orange, A. (reporteur). (2015). Brûlures gastriques : des médicaments pas si innocents [Reportage]. Dans F. Clément (réalisatrice), 36.9$^{\circ}$. Genève, Suisse : Radio Télévision Suisse. \\
 \vspace*{1cm} \\
{\huge{\textbf{Figures}}}

\bibitem{kldsfjds}
Figure 1. \textit{Lieu d’action de différentes classes d’antiacides.}

\bibitem{sdlkgsf}
Figure 2. \textit{Mécanisme, délai, durée et site d’action des trois différents médicaments}

\bibitem{wewqr}
Figure 3. \textit{Utilisation des trois médicaments en fonction de la fréquence (symptômes saltuaires     vs. Chroniques) et de la sévérité du trouble}

\bibitem{aaa}
Figure 4. \textit{Les pôles de transfert des connaissances scientifiques à la pratique médicale}

\bibitem{sdfdsg}
Figure 5. \textit{Représentation schématique de la théorie du comportement planifié de Fishbein \& Ajzen (1975-1991)}
\end{thebibliography}

